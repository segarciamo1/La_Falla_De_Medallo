\documentclass{article}
\usepackage[utf8]{inputenc}
\usepackage[spanish]{babel}
\usepackage{listings}
\usepackage{graphicx}
\graphicspath{ {images/} }
\usepackage{cite}

\usepackage[T1]{fontenc}

\usepackage{times}

\usepackage{color}
\definecolor{gray97}{gray}{.97}
\definecolor{gray75}{gray}{.75}
\definecolor{gray45}{gray}{.45}


\lstset{ frame=Ltb,
framerule=0pt,
aboveskip=0.5cm,
framextopmargin=3pt,
framexbottommargin=3pt,
framexleftmargin=0.4cm,
framesep=0pt,
rulesep=.4pt,
backgroundcolor=\color{gray97},
rulesepcolor=\color{black},
%
stringstyle=\ttfamily,
showstringspaces = false,
basicstyle=\small\ttfamily,
commentstyle=\color{gray45},
keywordstyle=\bfseries,
%
numbers=left,
numbersep=15pt,
numberstyle=\tiny,
numberfirstline = false,
breaklines=true,
}

% minimizar fragmentado de listados
\lstnewenvironment{listing}[1][]
{\lstset{#1}\pagebreak[0]}{\pagebreak[0]}

\lstdefinestyle{consola}
{basicstyle=\scriptsize\bf\ttfamily,
backgroundcolor=\color{gray75},
}

\lstdefinestyle{C++}
{language=C++}

\begin{document}

\begin{titlepage}
    \begin{center}
        \vspace*{1cm}
            
        \Huge
        \textbf{Proyecto Final: La Falla De Medallo}
            
        \vspace{0.5cm}
        \LARGE
        Informa 2 S.A.S.
            
        \vspace{1.5cm}
            
        \textbf{Juan José Baquero Arcila\\
                Sebastián García Morales}

        \vfill
            
        \vspace{0.8cm}
            
        \Large
        Departamento de Ingeniería Electrónica y Telecomunicaciones\\
        Universidad de Antioquia\\
        Medellín\\
        Abril 05 de 2022
    \end{center}
\end{titlepage}

\tableofcontents

\newpage
\section{Objetivos}\label{objetivos}
\begin{itemize}
    \item Aplicar los conocimientos adquiridos en el curso de informática II tanto en la parte teórica como en los laboratorios para diseñar un videojuego aplicando sistemas físicos en este.
    \item Practicar el uso de la interfaz gráfica de QT y añadir recursos como imágenes, sonidos y texturas al juego.
    \item Desarrollar habilidades de lógica y creatividad para realizar los diseños a ser implementados en el juego.
    
\end{itemize}
\section{Introducción}\label{intro}
El terremoto más fuerte en la historia de Colombia empezó, y tú estás estudiando en la universidad, ¿qué pasará? En la falla de Medellín queremos mostrarte como es que pensamos que podría ser este escenario en una ciudad como Medellín.\\

Acá en la falla tendrás que pasar por diferentes lugares para poder salvar tu pellejo de la inminente destrucción que se avecina, corre horizontalmente para escapar del desastre evitando obstáculos y enemigos que cambian dependiendo del lugar en que te encuentres; mientras vas recogiendo víveres que te ayudaran a sobrevivir a esta catástrofe.


\section{Descripción del juego:}\label{descripcion}

\subsection{La Falla De Medellín.}

El juego consistirá en un hipotético terremoto de 15 de magnitud en la escala de Richter en Medellín. Constará de tres niveles, en los cuales tendremos diferentes personas que se enfrentan a la situación de este terremoto en diferentes lugares, esquivando diferentes obstáculos y enfrentando enemigos durante el desastre que dejaría este. Los lugares serían la universidad, después el centro de la ciudad y por último sería el campo escapando de todo mal.\\

En este juego se tendrá:

\begin{itemize}
    \item \textbf{Plataformas:} estas tendrán varias utilidades en el juego, un as de estas plataformas el jugador podrá estar quito mirando donde saltar, otras estarán en el escenario, pero en unos segundos después de que el personaje se posicione encima de ellas, se precipitaran al piso o al cráter formado por la falla, también servirán en algunas ocasiones de obstáculo que caen en forma de escombros desde el cielo en diferentes trayectorias, también se tendrán algunas plataformas especiales que servirán de trampolín para pasar a otros lugares que normalmente no se podrían con un salto normal.
    \item \textbf{Obstáculos:} En el primer escenario se tendrán llantas que se cayeron de los carros, las cuales estarán deambulando por toda la universidad tratando de golpear a los jugadores y estos tendrán que esquivarlos.
    \item \textbf{Enemigos:} En el segundo escenario, que será ambientado en el centro de Medellín, se tendrán ladrones, ya que gracias al terremoto, las cárceles y demás centros de reclusión, fueron fuertemente afectados, y los que sobrevivieron están en las calles atracando y felices por su pronta liberación, estos lanzaran proyectiles como rocas (movimiento parabólico) para robar las pertenecías al personaje, ya que este tendrá un bolso con los útiles necesarios para el estudio como cuadernos, lapiceros, reglas, calculadoras, etc.\\
    En el campo habrá aves que te atacan porque están asustadas con movimientos circular uniforme y vacas que están descontroladas irán hacia el personaje para quitarle las vidas.
\end{itemize}

\subsection{Sistema de Vidas}
Se tendrá un sistema de 3 vidas, las cuales serán reducidas de a una a medida que los obstáculos, plataformas, enemigos y/o proyectiles intervengan. Por ejemplo, si se cae de la plataforma o le cae un escombro, si una vaca o un pájaro te tocan también perderás y si un enemigo te ataca con un proyectil.

\subsection{Funcionamiento de recompensas y cronometro.}

El juego tendrá un sistema de tiempo y de puntuación, los cuales intervendrán en el puntaje final de los participantes. La puntuación funcionará cogiendo objetos que te servirán en un hipotético terremoto y subirán tu puntaje. Las bonificaciones serian:

\begin{itemize}
    \item Pizzas para la universidad.
    \item Latas de agua para el centro de Medellín
    \item Huevos para la granja o campo.
\end{itemize}

Todos estos dando de a 10 puntos por cada objeto conseguido.\\
El tiempo nos dará un puntaje al final que nos sumará puntaje según el rango de tiempo en el que quede.

\subsection{Potenciadores.}

También se tendrán ciertos potenciadores que nos ayudarán a sobrepasar los niveles, tales como trampolines para pasar precipicios y diferentes vidas repartidas por el mapa para ayudar al participante.

\subsection{Distribución del juego.}
El juego será distribuido horizontalmente, es decir, nos vamos a estar movimiento más que todo en el eje X, variando con los objetos ya mencionados anteriormente. También tendremos un sistema de checkpoint, el cual será a mitad de los escenarios y servirá para cargar, guardar partida y salirse de esta.

\section{Descripción de objetos del juego:}\label{DescObjetos}

Para hablar del juego tenemos que mencionar las diferentes características y métodos que nuestros objetos en escena van a tener y como interactúan entre ellos, Así como se construye el escenario y como se va moviendo. Esto nos servirá para la hora de codificar, tener una idea más clara de lo que queramos que nuestros objetos tengan.

\subsection{Personaje principal.}

Esta será la clase encargada de generar el personaje principal, el cual será un personaje en 2D que será un estudiante común de la universidad de Antioquia. Los atributos que tendrán serán las posiciones (X, Y) de este en la escena, su tamaño en esta, velocidades en los dos ejes (Vx = 0 Vy = caída libre o MOV parabólico) y las vidas de este.\\
Contará con los métodos para moverse a través de los escenarios, estos serán con la ayuda de las teclas W para ir para arriba, A para ir a la izquierda, D para ir a la derecha y la tecla S para agacharse.

\subsection{Plataformas.}

Esta clase será encargada de crear los escombros que nos caen del cielo y de los bloques que se caen del escenario principal que recorre el jugador (suelo que se cae). Este contará con atributos tales como posición (X, Y), tamaño, durabilidad, velocidad en el eje ‘Y’ como caída libre. Estos atributos nos servirán para utilizarlo en el método de movimiento o de caer, este será en el eje y es para el suelo que se desploma. También tendremos parabólico o caída libre para los obstáculos que caen desde el cielo y resortes que nos ayudan a saltar escenarios. 

\subsection{Enemigos.}

En la clase enemigos tendremos dos tipos de enemigos, los cuales serán los encargados de hacernos perder vidas y no dejarnos pasar fácilmente a través del escenario. Contarán con atributos tales como posición (X, Y), tamaño, velocidades en X y Y.\\

Sus métodos serán moverse por el escenario en caso del campo con los pájaros en MCU o como las vacas o llantas que van de lado a lado en el escenario. Los ladrones tendrán un método para disparar proyectiles y estos serán en movimiento parabólico.   

\subsection{Escenario principal.}

Acá se desarrollarán todas las interacciones entre objetos, acá se tendrán varias interfaces, las cuales serán la de inicio de partida, la del juego en sí y la de cargar partida e interfaces.\\

Los escenarios para trabajar en el juego serian:
\begin{itemize}
    \item \textbf{La universidad de Antioquia}, en esta el estudiante tendrá que atravesar la universidad para poder escapar del desastre que este terremoto este está generando que se caigan los diferentes edificios, los vehículos explotan y sus llantas vuelan alrededor del campus.
    \item \textbf{El centro de la ciudad de Medellín}, en este, estarás con los peligros que el centro trae, tales como mayor cantidad de edificios que se desmoronan, ladrones que se volaron de las cárceles que quieren acabar contigo y llantas de vehículos que se destruyeron.
    \item Por último, tendremos \textbf{el campo}, en estos escenarios tendremos que pasar ya casi a las afueras de la ciudad para estar a salvo de este gran desastre en el que las calles se parten. Acá tendremos que esquivar diferentes animales descontrolados gracias al desastre natural que está pasando. Tendremos que esquivar vacas y hordas de pájaros que nos atacaran por lo confundidos que están.
\end{itemize}
\end{document}
